%! suppress = EnDash
%! suppress = MissingLabel
%! suppress = UnresolvedReference
%! suppress = NonMatchingIf
%% E-LIS-AST-MET-1012.tex
%%
%% Front page for Critical algorithms.
%%
%% 2023-07-27: Version for FDR.
%%
\newcommand{\doctitle}{%
  \parbox[t]{\textwidth}{Software Modules,\\Simulated Data, and\\Documentation Package}}
\newcommand{\docnumber}	{E-LIS-AST-MET-1012}     % Document number
\newcommand{\issuenumber}{1-0}                   % Version
\newcommand{\issuedate}{2023-07-27}              % Date
\newcommand{\workpackage}{8.2}                   % Workpackage

\newcommand{\authorname}{%                       % Authors
  T.~Marquart,\\
  G.~Otten,\\
  N.~Sabha,\\
  M.~Baláž,\\
  O.~Czoske,\\
  H.~Buddelmeijer,\\
  K.~Leschinski,\\
}
\newcommand{\authorsigndate}  {2023-06-31} % date of signature of author

\newcommand{\reviewername}     {%
}       % checked by name
\newcommand{\reviewersigndate} {} % date of signature of checker

\newcommand{\approvername}    {%
}       % name of the approver
\newcommand{\approvalsigndate}{} % date of signature of approver

\newcommand{\releasername}{%
}
\newcommand{\releasesigndate}{}  % date of release

% Title displayed in the header
\newcommand{\shorttitle}{Software Modules, Simulated Data, and Documentation Package}


%%%%%%%%%%%%%%%%%%%%%%%%%%%%%%%%%%%%%%%%%%%%%%%%%%%%%%%%%%%%%%%%%%%%%%%%%%%%%
\documentclass[11pt,oneside,a4paper]{article}

%% Layout  --- TeX commands
\raggedbottom
\topmargin=-9mm
\headsep=0.5cm
\textheight=246mm
\textwidth=164mm
\hoffset=0cm
\oddsidemargin=-2mm
\parindent=0mm
\parskip=0.3em

\renewcommand{\floatpagefraction}{0.98} % separate float page does not work
\renewcommand{\bottomfraction}{1.0}

%% Fonts
\usepackage{mathptmx}
\usepackage[scaled=0.92]{helvet}

\usepackage[pdftex]{graphicx}
\graphicspath{{./figures/}}
\renewcommand{\bottomfraction}{1.0}

\usepackage{fancyhdr}
\usepackage{titlesec}
\usepackage{booktabs}
\usepackage{tabularx}
\usepackage{array}
\usepackage{longtable}
\usepackage{numprint}
\usepackage{placeins}
\usepackage{enumitem}
\usepackage{acronym}
\usepackage{filecontents}

\usepackage{xcolor}
\usepackage{colortbl}
\definecolor{listingbg}{gray}{0.95}
\definecolor{darkgreen}{rgb}{0.0, 0.7, 0.0}
\definecolor{darkblue} {rgb}{0.0, 0.0, 0.7}
\definecolor{darkred}  {rgb}{0.7, 0.0, 0.0}
\definecolor{darkorange}{rgb}{1.0, 0.49, 0.0}

\usepackage[many]{tcolorbox}

% does not work with my version of biblatex:
% \usepackage[defernumbers=true,backend=biber,hyperref=true,url=false]{biblatex}
\usepackage[%
   defernumbers=true,
   backend=biber,
   hyperref=true,
   url=false,
   sorting=none]{biblatex}

\usepackage[%
% pagebackref=true,   % does not work with biblatex
   pdfpagelabels=true,
   plainpages=false,
   colorlinks=false]{hyperref}

% does not work with my version:
%\usepackage[xindy,nopostdot,nogroupskip]{glossaries}
\usepackage[xindy]{glossaries}
%\makeglossaries

\usepackage{lastpage}
\usepackage[%
   font=small,
   format=plain,
   indention=1.5em,
   labelfont={small,bf},
   up,
   justification=justified,
   singlelinecheck=true]{caption}

%% for sidewaysfigure and sidewaystable
\usepackage{rotating}

%% put recipe names, QC parameters, etc., in \lstinline{}
\usepackage{listings}
\lstloadlanguages{csh, c}

\usepackage{textcomp}

\usepackage{subfloat}

%% Write requirements like this: \REQ{METIS-xxxx}
\newcommand{\REQ}[1]{\href{https://polarion.astron.nl/polarion/\#/project/METIS/workitem?id=#1}{\textcolor{brown}{#1}}}

%% Write code snippets like this: \CODE{SOMETHING==ELSE}
\newcommand{\CODE}[1]{\lstinline[]!#1!}

%% Bright red for things that still need to be looked at. None of
%% these should appear in the release version
\newcommand{\TODO}[1]{\textcolor{red}{\bfseries TODO: #1}}

%=== Header / Footer definition ===========================
\newcommand{\headerformat}{%
  \lhead{\small\sffamily
    \begin{tabularx}{\textwidth}{Xlll}
      \shorttitle & \docnumber & \issuenumber & \issuedate\\[0.7ex]
    \end{tabularx}}
  \rhead{}
  \cfoot{\small\bfseries\sffamily
    Page {\textbf{\thepage}}  of {\textbf{\pageref{LastPage}}} }
  \rfoot{\raisebox{-0.3\height}{\includegraphics[width=25mm]{metis_noText}} }
}

\fancypagestyle{plain}{\fancyhf{} \headerformat}
\headerformat
\definecolor{brn}{RGB}{99,36,35}
\definecolor{rd1}{RGB}{229,184,183}
\definecolor{rd2}{RGB}{242,219,219}

\renewcommand\headrule{%
  \begingroup
  \color{brn}
  \hrule height 0.5pt width\headwidth
  \endgroup
}

\renewcommand{\contentsname}{Table of Contents}

%% ---- Format of section titles --------------------------------
% Combining sffamily and scshape works only for some fonts
\titleformat{\section}{\LARGE\sffamily\scshape}{\thesection}{1em}{\textsc{}}
\titleformat{\subsection}{\Large\sffamily}{\thesubsection}{1em}{}
\titleformat{\subsubsection}{\large\sffamily}{\thesubsubsection}{1em}{}

\setlength\heavyrulewidth{1.5pt}


% === PDF Definition ==========================================================
\hypersetup{%
  setpagesize,
  bookmarksnumbered=true,
  pdfborder= 0 0 0,
  pdftitle={\shorttitle},
  pdfauthor={\authorname}}


% === biblatex Definitions ====================================================
\begin{filecontents}{references.bib}
@Manual{DRLD,
  title =        {METIS Data Reduction Library Design},
  version =      {1-0},
  number =       {E-REP-AST-MET-1006},
  keywords =     {applicable},
  note =         {2023-09-18}
  }
\end{filecontents}

\bibstyle{apj}
\addbibresource{references.bib}


% === glossaries definitions  =================================================
\renewcommand{\glossarysection}[2][]{}   % Avoid the heading 'Glossary'
\setlength{\LTleft}{0pt}                 % make all longtables aligned left
\setlength{\glsdescwidth}{0.8\textwidth}


%%%%%%%%%%%%%%%%%%%%%%%%%%%%%%%%%%%%%%%%%%%%%%%%%%%%%%%%%%%%%%%%%%%%%%%%%%%%%
%
% Start of document
%
%%%%%%%%%%%%%%%%%%%%%%%%%%%%%%%%%%%%%%%%%%%%%%%%%%%%%%%%%%%%%%%%%%%%%%%%%%%%%
%
\begin{document}

% listings style
\lstset{basicstyle=\ttfamily\lst@ifdisplaystyle\scriptsize\fi,
  columns=flexible,
  frame=single,
  backgroundcolor=\color{listingbg},
  captionpos=b,
  showspaces=false}


%%%%% titlepage
\thispagestyle{empty}

% Logo
\vspace*{0cm}
\includegraphics[width=6.69cm]{metis_logo}

\vspace*{\fill}

% Document title
{\color{brn}\rule[1.9ex]{\textwidth}{1.5pt}}
\scalebox{1.44}{\Huge\textsf{\doctitle}}\\
{\color{brn}\rule{\textwidth}{1.5pt}}\\ [0.5ex]

% Document info
{\Large\textsf{\docnumber}}  \\ [1ex]
{\Large\textsf{Issue \issuenumber}} \\ [1ex]
{\Large\textsf{\issuedate}}  \\ [1ex]
{\Large\textsf{Work package: \workpackage}}  \\[1ex]

\vspace*{\fill}

% Signature table
\begin{center}
  \renewcommand{\arraystretch}{0.75}
%  \begin{tabularx}{\textwidth}{XXXX}
  \begin{tabular}{p{0.25\textwidth}p{0.25\textwidth}p{0.25\textwidth}p{0.25\textwidth}}
    \arrayrulecolor{brn}
    \toprule
    & \multicolumn{3}{l}{\scriptsize\textsf{Signature and Approval}} \\
    \midrule
    & {\scriptsize\textsf Name}
    & {\scriptsize\textsf Date}
    & {\scriptsize\textsf Signature} \\
    \cline{2-4}
    \\
    \textsf{Prepared} & \parbox[c]{\hsize}{\raggedright \authorname} & \authorsigndate & \\
    \\
    \midrule
    \\
    \textsf{Reviewed} & \parbox[c]{\hsize}{\raggedright \reviewername} & \reviewersigndate & \\
    \\
    \midrule
    \\
    \textsf{Approved} & \parbox[c]{\hsize}{\raggedright \approvername} & \approvalsigndate & \\
    \\
    \midrule
    \\
    \textsf{Released} & \parbox[c]{\hsize}{\raggedright \releasername} & \releasesigndate & \\
    \\
    \bottomrule
     %    \end{tabularx}
  \end{tabular}
\end{center}


% === MAIN TEXT================================================================

\clearpage
\pagestyle{fancy}

% === Revision History =========================================================
\section*{Revision History}
\renewcommand{\arraystretch}{1.2}
\arrayrulecolor{brn}
\begin{tabularx}{\textwidth}{|l|l|l|X|}
  \hline
  \rowcolor{rd1}
  \textbf{Issue} & \textbf{Date} & \textbf{Owner} & \textbf{Changes} \\
  \hline
  1-0 & 2023-07-27 & K. Leschinski & FDR release\\
  \hline
\end{tabularx}


%=== Indexes ==================================================================
%\newpage
%\tableofcontents
%\clearpage
%\listoffigures
%\clearpage
%\listoftables

%\clearpage
%\phantom{a}
%\vfill
%\begin{center}
%  This page intentionally left blank
%\end{center}
%\vfill
%\clearpage

%%%%%%%%%%%%%%%%%%%%%%%%%%%%%%%%%%%%%%%%%%%%%%%%%%%%%%%%%%%%%%%%%%%%%%%%%%%%%


% INTRODUCTION
\section{Introduction}
\label{sec:intro}

\subsection{Applicable documents}
\label{ssec:ad}
\begin{refcontext}[labelprefix=AD]
  \printbibliography[keyword=applicable, heading=none]
\end{refcontext}

\subsection{Scope}
\label{ssec:scope}
The critical algorithms as described in section 10 of the METIS Data Reduction Library Design~\cite{DRLD} are stored in a git repository at \url{https://github.com/AstarVienna/CriticalAlgorithms}.

Ten algorithms were deemed critical at the Preliminary Design Review (PDR).
Only four of the algorithms are still considered critical and require prototyping.
The rationale for which algorithms are critical is given in~\cite{DRLD}.

The four algorithms that are prototyped are the following:
\begin{enumerate}
    \item[5a.] Long Slit Spectroscopy (LSS) Wavelength calibration and distortion correction.\\
        See directory \CODE{5a-LSS-wavelength-calibration-PyReduce-ELT}.
    \item[5b.] Integral Field Unit (IFU) Wavelength calibration and distortion correction.\\
        See directory \CODE{5b-IFU-distortion}.
    \item[9.] IFU image and cube reconstruction.\\
        See directory \CODE{9-IFU-image-reconstruction}.
    \item[11.] Angular Differential Imaging algorithm.\\
        See directory \CODE{11-ADI-algorithm}.
\end{enumerate}


\section{Installation}
\label{sec:installation}
This section describes how to clone the repository.
The instructions for running the prototypes are given in their respective directories and in~\cite{DRLD}.

\subsection{Git LFS}
\label{ssec:gitlfs}
The repository uses git-lfs and git-submodules.
It is therefore necessary to install git-lfs.
This has to be done only once, and might already be done for your setup.

First install git-lfs, which depends on your specific system.
For example on Ubuntu do
\begin{lstlisting}
sudo apt-get install git-lfs
\end{lstlisting}

Then initialize git-lfs:
\begin{lstlisting}
git lfs install
\end{lstlisting}

\subsection{Cloning the repository}
\label{ssec:cloning}

The repository can be cloned once git-lfs is installed and initialized:

\begin{lstlisting}
git clone --recurse-submodules https://github.com/AstarVienna/CriticalAlgorithms.git
\end{lstlisting}


%\subsection{Reference documents}
%\label{ssec:rd}
%\begin{refcontext}[labelprefix=RD]
%  \printbibliography[keyword=reference, heading=none]
%\end{refcontext}
%\clearpage

%\subsection{Acronyms}
%\label{ssec:acronyms}
%\input{acronyms.tex}
%\clearpage

\end{document}

%%
%% THE END
%%
%%%%%%%%%%%%%%%%%%%%%%%%%%%%%%%%%%%%%%%%%%%%%%%%%%%%%%%%%%%%%%%%%%%%%%%%%%%%%
